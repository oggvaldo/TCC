Em um mundo onde os avanços tecnológicos são tão rápidos, a tomada de decisão para uma ágil implantação e/ou adaptação se torna também necessária, levando a criação de várias metodologias que vão auxiliar desde a idealização, passando pelo planejamento e implantação até a finalização destes projetos que foram pensados. Com este desafio se aplicando a todo o mundo, não seria impensável que mesmo no ramo de Inteligência Artificial o modelo de desenvolvimento de software ágil também estivesse incluso. Porém, neste modelo também vemos a necessidade de pensar nas implicações éticas do uso da tecnologia, que traz consigo importantes desafios a serem superados. Neste trabalho, é discutido como a aplicação da metodologia ágil impactou na criação de um guia para auxiliar os desenvolvedores e usuários finais na implantação de princípios éticos no contexto de aplicações de Inteligência Artificial (IA). Além disso, é apresentado como foi o desenvolvimento o guia proposto e seu devido uso, gerando assim um impacto imediato para os potenciais usuários e possibilidades de melhoras futuras, tanto nos projetos futuros quanto no projeto da própria ferramenta.


\todo[inline]{acho que no resumo você precisa: contextualizar, colocar o objetivo do trabalho, o método usado, os resultados e as conclusões -- de maneira breve, no máximo uma pagina para tudo isso -- Veja com o José como vocês podem colocar nessa estrutura.}