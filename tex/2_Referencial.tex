\label{referencial}

Neste Capítulo apresentamos ...

\section{Engenharia de Software}
 Engenharia de software é o processo que foca nas áreas de planejamento, desenvolvimento e entrega dos sistemas de software. É com este tipo de metodologia que conseguimos estimar de maneira mais apurada formas de como desenvolver o sistema, estimar prazos, recursos e até mesmo espaços para melhora durante e após a conclusão do projeto. A engenharia de software nasceu da necessidade de entregar ao cliente uma maior garantia de qualidade, sem deixar de lado a preocupação com prazos, estes cada vez mais curtos perante as demandas que o mundo e sua evolução tecnológica tem exigido \cite{DBLP:books/lib/Sommerville07}. 

\subsection{Engenharia de Requisitos}
A engenharia de requisitos (ou especificação de software) é a área responsável por entender e decidir quais serão as requisições necessárias ao sistema solictado e verificar os impedimentos relacionados ao desenvolvimento e operação do sistema. Costuma ser um estágio crítico do processo de criação de um software, pois uma vez mal desenhado e com erros nesta fase, mais a frente no projeto, implantação e manutenção do sistema problemas serão comuns de aparecerem \cite{DBLP:books/lib/Sommerville07}.


\section{Desenvolvimento Ágil de Software}

\subsection{Scrum}

\subsection{Planning Poker}

\section{Inteligência Artificial}

\subsection{Ética em IA}


\section{Trabalhos Relacionados}
