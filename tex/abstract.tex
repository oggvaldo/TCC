In a world where technological improvements are really quick, have the correct decision for a faster creation and/or adapt becomes more and more requested, asking for creating various methodologies which can help since the idealization, passing through the planning and implementing until the ending of those projects that were thought. With that challenge being applied for all the world, it would not be unthinkable that even on Artificial Intelligence area the agile software development were included. But, with this model we see how thinking on ethical implications using this technology becomes required to passthrough those challenges. On this work, we'll discuss how the agile methodology has impacted on the creation for a guide to help developers and stakeholders implementing those ethical principles at the Artificial Intelligence context. Besides that, it is presented how was the development for this guide and using the ECCOLA methodology, created by Vakkuri et al. \cite{ECCOLA} as a proof of concept for this tool, generating an immediate impact for potential users. We also show how to modify this tool to use with other methodologies with other parameters, user cases and future improvements, both on future projects and on this tool development.


\todo[inline]{Deixe o abstract para traduzir após fechar o resumo --- Se vc tiver dificuldades com o Ingles acho que o José pode ajudá-lo.}
Em um mundo onde os avanços tecnológicos são tão rápidos, a tomada de decisão para uma ágil implantação e/ou adaptação se torna também necessária, levando a criação de várias metodologias que vão auxiliar desde a idealização, passando pelo planejamento e implantação até a finalização destes projetos que foram pensados. Com este desafio se aplicando a todo o mundo, não seria impensável que mesmo no ramo de Inteligência Artificial o modelo de desenvolvimento de software ágil também estivesse incluso. Porém, neste modelo também vemos a necessidade de pensar nas implicações éticas do uso da tecnologia, que traz consigo importantes desafios a serem superados. Neste trabalho, é discutido como a aplicação da metodologia ágil impactou na criação de um guia para auxiliar os desenvolvedores e usuários finais na implantação de princípios éticos no contexto de aplicações de Inteligência Artificial (IA). Além disso, é apresentado como foi o desenvolvimento de um guia e seu devido uso usando como prova de conceito o método ECCOLA, criado por Vakkuri et al. \cite{ECCOLA}, gerando assim um impacto imediato para os potenciais usuários. Mostramos também como modificar o guia para uso com outros parâmetros desenvolvidos, casos de uso e possibilidades de melhoras futuras, tanto nos projetos futuros quanto no projeto da própria ferramenta.