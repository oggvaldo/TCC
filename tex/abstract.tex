Em um mundo onde os avanços tecnológicos são tão rápidos, a tomada de decisão para uma ágil implantação e/ou adaptação se torna também necessária, levando a criação de várias metodologias que vão auxiliar desde a idealização, passando pelo planejamento e implantação até a finalização destes projetos que foram pensados. Com este desafio se aplicando a todo o mundo, não seria impensável que mesmo no ramo de Inteligência Artificial este modelo de desenvolvimento ágil também estivesse incluso. Porém, também junto com este desenvolvimento, vemos a necessidade de pensar nas implicações éticas do uso de tal tecnologia, que traz consigo importantes desafios a serem superados. Neste trabalho, veremos como a aplicação da metodologia ágil impactou na criação de um framework para auxílio dos desenvolvedores e usuários da tecnologia a facilitar o debate e implantação de princípios éticos de IA em seus trabalhos, também como foi o desenvolvimento da ferramenta e seu uso devido, gerando assim um impacto imediato para os potenciais usuários e possibilidades de melhoras futuras, tanto nos projetos futuros quanto no projeto da própria ferramenta.