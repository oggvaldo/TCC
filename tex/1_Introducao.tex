\label{introducao}
%\section{Normas CIC}
% \href{http://monografias.cic.unb.br/dspace/normasGerais.pdf}{Política de Publicação de Monografias e Dissertações no Repositório Digital do CIC}%
% \href{http://monografias.cic.unb.br/dspace/}{Repositório do Departamento de Ciência da Computação da UnB}

% \href{http://bdm.bce.unb.br/}{Biblioteca Digital de Monografias de Graduação e Especialização}

A cada dia que passa, temos um mundo onde as tomadas de decisões se tornam mais e mais cruciais, assim como temos um mundo que tem feito da área da \acrshort{TI} mais e mais área base para outras competências, seja desde organização de estoques até a área de computação ubíqua, passando pelo uso de mais e mais gadgets com um imenso poder computacional, como smartphones, tablets, smart TVs, smartwatches, assistentes pessoais, entre outros. Em um mundo com tanta dependência computacional como hoje em dia, tem se tornado mais e mais comum o advento do uso de técnicas de computação baseadas em aprendizado de máquina (\textit{Machine Learning}), análise de dados e \acrshort{IA}. Porém junto com este uso massivo de tais áreas, começamos a ver fortes implicações sociais nas vidas das pessoas, como, por exemplo, a forma de determinar como e até que ponto a \acrshort{IA} se encontrará presente em um determinado sistema ou  


\section{Problema de Pesquisa}


\section{Justificativa}


\section{Objetivos}


\subsection{Objetivo Geral}


\subsection{Objetivo Específicos}


\section{Metodologia de Pesquisa}


\section{Estrutura do Trabalho}
Na Seção \ref{desenvolvimento}.

%Engenharia de Software, Agil, Scrum, Planning Poker, IA, Engenharia de Requisitos, Engenharia de Requisitos para IA