O advento do ensino a distância no Brasil vem em uma crescente maior e maior a cada dia, desde sua gênese no ensino profissional formal através do ensino com cartas, onde no país a primeira instituição educacional a realizar tais cursos foi o Instituto Universal, a partir da década de 40 ~\cite{instituto_universal}, e que se mantém relevante até hoje ensinando a várias pessoas através de seus módulos seja pela internet ou, até hoje e ainda, por carta. 

É uma realidade que faz parte da história educacional brasileira o ensino a distância, onde não em vão, ajuda até hoje milhões de brasileiros a terem acesso a educação formal e até ao seu diploma, seja do ensino básico, seja do ensino profissional ou até do ensino superior, este último podendo ser tanto graduação quanto pós graduação. 

Com tamanha importância e avanço da tecnologia, era questão de tempo até que tal ensino pudesse ser popularizado para acessar a grande rede, e assim termos a necessidade do uso de sistemas especializados para a plataforma. Juntamente com o uso dos sistemas, também temos problemas que afligem o ensino como um todo, como manter o interesse do aluno ao longo do curso, fazê-lo entender a ideia da matéria, saber criticar e pensar em cima da matéria, desafios que a educação tem encarado todos os dias e necessitando se reinventar conforme o tempo, situações e suas novas tecnologias apresentam como possíveis obstáculos na dificuldade do aprendizado. 

Um deles, com o foco no EAD, a adaptabilidade do aluno de acordo seu nível de pré-conhecimento do curso, ou até mesmo alta capacidade de absorção intuitiva da matéria. Com um aluno que conhece previamente ou tem uma alta capacidade de resposta aos módulos, aquele curso pode se tornar desmotivante por ter que repassar um conhecimento o qual já tem há tempos, ou que o aluno julga muito fácil. Com um aluno que tem muita dificuldade em absorver a matéria, o mesmo pode ser desmotivante, por não possuir as bases necessárias para se manter em movimento ao longo do curso e, ainda assim, ver que tem a capacidade de aprender aquele módulo, por mais desafiante que seja. 

Exatamente com este tipo de desafio em mente, a proposta deste trabalho é criar um sistema capaz de nivelar automaticamente o aluno de acordo sua capacidade de aprendizado, pondo-o em um ritmo que irá manter o momento de aprendizado em mais alta possível, porém sempre respeitando as capacidades do aluno. Ao final deste trabalho, não apenas a ideia será maturada para desenvolvimento, mas um protótipo funcional, que pode mostrar na prática como esta ferramenta poderá auxiliar e revolucionar na capacidade do ensino a distância para o futuro.

%\url{http://www.escritacientifica.com/}

%%%%%%%%%%%%%%%%%%%%%%%%%%%%%%%%%%%%%%%%%%%%%%%%%%%%%%%%%%%%%%%%%%%%%%%%%%%%%%%%
%%%%%%%%%%%%%%%%%%%%%%%%%%%%%%%%%%%%%%%%%%%%%%%%%%%%%%%%%%%%%%%%%%%%%%%%%%%%%%%%
%%%%%%%%%%%%%%%%%%%%%%%%%%%%%%%%%%%%%%%%%%%%%%%%%%%%%%%%%%%%%%%%%%%%%%%%%%%%%%%%
\section{História do \acrfull{EAD}}%
Começar com as ideias retiradas através do site https://www.ead.com.br/ead/como-surgiu-ensino-a-distancia.html, pincelando um pouco do ensino a distância na humanidade e Brasil e como se deu o desenvolvimento do mesmo no país, procurando dar foco no quanto o ensino foi moldado ao longo do tempo e como ele foi capaz de ajudar a milhões de brasileiros a chegar no objetivo de possuir um diploma. Ressaltar não somente aspecto social, porém também educacional e como é uma ferramenta que o espaço para o crescimento será absurdamente alto ao longo do tempo.

%%%\\acrfull{sigla} é o comando para usar a abreviação.

%%%%%%%%%%%%%%%%%%%%%%%%%%%%%%%%%%%%%%%%%%%%%%%%%%%%%%%%%%%%%%%%%%%%%%%%%%%%%%%%
%%%%%%%%%%%%%%%%%%%%%%%%%%%%%%%%%%%%%%%%%%%%%%%%%%%%%%%%%%%%%%%%%%%%%%%%%%%%%%%%
%%%%%%%%%%%%%%%%%%%%%%%%%%%%%%%%%%%%%%%%%%%%%%%%%%%%%%%%%%%%%%%%%%%%%%%%%%%%%%%%
\section{Desafios edicacionais do \acrfull{EAD}}%
Escrever sobre os desafios educacionais presentes no Ensino a distância, com um histórico em como problemas de atenção nos alunos, em especial nos cursos presenciais, podem ser um problema. Ressaltar o problema do ensino de uma matéria não interessante ao aluno, e como aulas não-presenciais podem potencializar este problema no aluno, por mais que o EAD permita ao aluno o estudo em seu tempo.


\begin{enumerate}
	\item exemplo1
	\item exemplo 2
	\item exemplo 3
\end{enumerate}%

Uma abordagem para esta metodologia é seguir os seguintes passos:
\begin{description}
	\item[Caracterização do Problema:] Qual a pergunta a ser respondida? Quais
informações/recursos necessários na investigação?
	\item[Formulação da Hípotese:] Quais explicações possíveis para o que foi observado?
	\item[Previsão:] Dadas explicações [corretas] para as observações, quais os
	resultados previstos?
	\item[Experimentos:] \ \\\vspace{-2em}
		\begin{enumerate}
			\item Execute testes [reproduzíveis] da hipótese, coletando dados.
			\item Analise os dados.
			\item Interprete os dados e tire conclusões:
				\begin{itemize}
				\item que comprovam a hipótese;
				\item que invalidam a hipótese \emph{ou levam a uma nova hipótese}.
				\end{itemize}
		\end{enumerate}
	\item[Documentação:] Registre e divulgue os resultados.
	\item[Revisão de Resultados:] Validação dos resultados por outras pessoas
	[capacitadas].
\end{description}%

\subsection{Veja Também}
\begin{itemize}
	\item Google Acadêmico
		\\\url{http://scholar.google.com.br/}%
	\item ACM Digital Library
		\\\url{http://dl.acm.org/}%
	\item Portal \acrshort{CAPES}
		\\\url{http://www.periodicos.capes.gov.br/}%
	\item IEEE Xplore
		\\\url{http://ieeexplore.ieee.org/Xplore/home.jsp}%
	\item ScienceDirect
		\\\url{http://www.sciencedirect.com/}%
	\item Springer Link
		\\\url{http://link.springer.com/}%
\end{itemize}

Para buscar referências, \emph{The DBLP Computer Science Bibliography}\footnote{\url{http://dblp.uni-trier.de/}}
é um ótimo recurso. Veja o \refApendice{Apendice_Fichamento} para instruções
sobre como organizar as informações de artigos científicos.

%%\footnote{escrever aqui nota de rodapé}
%%\url{endereçodosite} cria um hyperlink

%%%%%%%%%%%%%%%%%%%%%%%%%%%%%%%%%%%%%%%%%%%%%%%%%%%%%%%%%%%%%%%%%%%%%%%%%%%%%%%%
%%%%%%%%%%%%%%%%%%%%%%%%%%%%%%%%%%%%%%%%%%%%%%%%%%%%%%%%%%%%%%%%%%%%%%%%%%%%%%%%
%%%%%%%%%%%%%%%%%%%%%%%%%%%%%%%%%%%%%%%%%%%%%%%%%%%%%%%%%%%%%%%%%%%%%%%%%%%%%%%%
%\section{\acrfull{TRI}}%

%Escrever nesta seção sobre o que é o Teorema de Resposta ao Item, o que ele é, como foi desenvolvida a fórmula matemática para o uso do mesmo, como ele funciona, exemplos de caso de uso no Brasil (ENEM) e mundo (TOEFL) e onde pode ser aplicado
\TeX\ é ``\emph{a typesetting system intended for the creation of beautiful books
 - and especially for books that contain a lot of mathematics}''~\cite{Knuth_1986_texbook},
 um sistema de tipografia muito utilizado na produção de textos técnicos devido
 a qualidade final, principalmente das fórmulas e símbolos matemáticos gerados.

\LaTeX\ é um conjunto de macros para facilitar o uso de \TeX~\cite{lamport_latex:_1994},
cujos pacotes (a maioria centralizada na rede {CTAN}~\cite{greenwade93}), oferecem
inúmeras possibilidades. Este sistema tipográfico visa explorar as potencialidades
da impressão digital, sem que o resultado seja alterado em função de diferenças
entre plataformas/sistemas.

Em uma publicação, um \emph{autor} entrega o texto a uma editor que define a
formatação do documento (tamanho da fonte, largura de colunas, espaçamento, etc.)
e passa as instruções (e o manuscrito) ao tipógrafo, que as executa. Neste processo,
\LaTeX\ assume os papéis de editor e tipógrafo, mas por ser ``apenas'' um programa
de computador, o autor deve prover algumas informações adicionais ~\cite{Oetiker_1995_notsoshort},
geralmente por meio de marcações (comandos).

Esta abordagem de linguagem de marcação (em que se indica como o texto deve ser
formatado) é diferente da abordagem OQVVEOQVO (``o que você vê é o que você
obtém\footnote{Do inglês WYSIWYG - ``What You See Is What You Get''.}'') de programas
para edição de texto tradicionais (como MS Word, LibreOffice Write, etc.).
Apesar destes programas serem extremamente úteis para gerar textos simples, que
são a grande maioria dos documentos, eles geralmente não têm a capacidade de lidar
corretamente com documentos complexos (como dissertações ou teses), conforme ilustrado
na \refFig{latexvsword}.%

\figuraBib{miktex}{\LaTeX\ vs MS Word}{pinteric_latex_2004}{latexvsword}{width=.45\textwidth}%

Existem diversas discussões quanto ao uso de editores de texto\footnote{Por exemplo:
\emph{Word Processors: Stupid and Inefficient} \url{http://ricardo.ecn.wfu.edu/~cottrell/wp.html}},
não há um consenso quanto a melhor forma de se gerar um documento de qualidade,
e a maioria das mídias científicas disponibiliza modelos para ambas.

Mas pode-se dizer que \LaTeX\ é mais indicado para:
\begin{itemize}
	\item notação matemática;
	\item referências cruzadas;
	\item separação clara entre conteúdo e formatação.
\end{itemize}

Enquanto os editores tradicionais são indicados para:
\begin{itemize}
	\item edição colaborativa (são mais populares);
	\item produção imediata (leve curva de aprendizado).
\end{itemize}

\subsection{Veja Também}
\begin{itemize}
	\item Introdução ao \LaTeX
		\\\url{http://latexbr.blogspot.com.br/2010/04/introducao-ao-latex.html}
	\item \LaTeX\ - A document preparation system
		\\\url{http://www.latex-project.org/}
	\item The \acrlong{CTAN}
		\\\url{http://ctan.org}
	\item \TeX Users Group
		\\\url{http://tug.org}
	\item \TeX\ - \LaTeX\ Stack Exchange
		\\\url{http://tex.stackexchange.com}
	\item \LaTeX\ Wikibook
		\\\url{http://en.wikibooks.org/wiki/LaTeX}
	\item write\LaTeX
		\\\url{http://www.writelatex.com}
\end{itemize}



%%%%%%%%%%%%%%%%%%%%%%%%%%%%%%%%%%%%%%%%%%%%%%%%%%%%%%%%%%%%%%%%%%%%%%%%%%%%%%%%
%%%%%%%%%%%%%%%%%%%%%%%%%%%%%%%%%%%%%%%%%%%%%%%%%%%%%%%%%%%%%%%%%%%%%%%%%%%%%%%%
%%%%%%%%%%%%%%%%%%%%%%%%%%%%%%%%%%%%%%%%%%%%%%%%%%%%%%%%%%%%%%%%%%%%%%%%%%%%%%%%
\section{Plágio}%
O JusBrasil\footnote{\url{http://www.jusbrasil.com.br}} define plágio como
``reprodução, total ou parcial, da propriedade intelectual de alguém, inculcando-se
o criador da idéia ou da forma. Constitui crime contra a propriedade imaterial
violar direito de autor de obra literária, científica ou artística.''


A \acrfull{CEP} da Presidência da República decidiu  ``pela
aplicação de sanção ética aos servidores públicos que incorrerem na prática de
plágio"\footnote{\url{http://www.comissaodeetica.unb.br/index.php?view=article&id=8:plagio-academico}},
e a \acrfull{CAPES} recomenda que se adote políticas de conscientização e informação
sobre a propriedade intelectual, baseando-se na Proposição 2010.19.07379-01,
referente ao plágio nas instituições de ensino\footnote{\url{https://www.capes.gov.br/images/stories/download/diversos/OrientacoesCapes_CombateAoPlagio.pdf}}.


\subsection{Veja Também}
\begin{itemize}
	\item IEEE Plagiarism FAQ
		\\\url{http://www.ieee.org/publications_standards/publications/rights/plagiarism_FAQ.html}
	\item Relatório da Comissão de Integridade de Pesquisa do CNPq
		\\\url{http://www.cnpq.br/web/guest/documentos-do-cic}
\end{itemize}



%\section{Normas CIC}
% \href{http://monografias.cic.unb.br/dspace/normasGerais.pdf}{Política de Publicação de Monografias e Dissertações no Repositório Digital do CIC}%
% \href{http://monografias.cic.unb.br/dspace/}{Repositório do Departamento de Ciência da Computação da UnB}

% \href{http://bdm.bce.unb.br/}{Biblioteca Digital de Monografias de Graduação e Especialização}