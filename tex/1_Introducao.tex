\label{introducao}

A cada dia que passa, temos um mundo onde as tomadas de decisões se tornam mais e mais cruciais, assim como temos um mundo que tem feito da área de Tecnologia da Informação e Comunicação (TIC) como uma área base para outras competências, seja desde a organização de estoques até a área de computação ubíqua, passando pelo uso de mais e mais dispositivos com um imenso poder computacional, como smartphones, tablets, smart TVs, smartwatches, assistentes pessoais, entre outros. 

Em um mundo com tanta dependência computacional como hoje em dia, tem se tornado mais e mais comum o advento do uso de técnicas de computação baseadas em aprendizado de máquina (\textit{Machine Learning}), análise de dados e Inteligência Artificial (IA). Porém, junto com este uso massivo de tais áreas, começamos a ver fortes implicações sociais na vida das pessoas, como, por exemplo, a forma de determinar como e até que ponto a \acrshort{IA} se encontrará presente em um determinado sistema. Concomitante com a situação contemporânea, é necessário que nos processos de implantação sejam postos em produção o quanto antes, o que hoje é possível graças as técnicas de metodologias ágeis que foram desenvolvidas desde a década de 70 e sendo formalmente manifestas em fevereiro de 2001 com o manifesto ágil \cite{agilemanifesto}. 

Com tais metodologias em voga, é natural que durante o desenvolvimento de projetos onde se envolve a \acrshort{IA} se depare com problemas, dilemas, dúvidas e discussões envolvendo a ética a ser abordada no projeto. E verificando a ausência de ferramentas que possam facilitar a solução deste problema, vemos neste projeto a oportunidade de preencher esta lacuna.



\section{Problema de Pesquisa}
Não há uma ferramenta digital que possa auxiliar times de desenvolvimento com o foco em \acrshort{IA}

\section{Justificativa}
Em um mundo onde os recursos estão sendo utilizados de forma mais racional a cada dia que se passa, é necessário haver uma ferramenta que possa ajudar aos desenvolvedores na área de \acrshort{IA} a adequarem também a parte ética, uma vez que com o advento desta área se começa a entrar em uma área cinza e ainda pouco desbravada.

\section{Objetivos}
Tem-se como objetivo a criação de uma ferramenta que possa ajudar os desenvolvedores a verificar tópicos escolhidos e debater sobre o assunto de forma lúcida, eficiente e acessível.

\subsection{Objetivo Geral}
%a preencher

\subsection{Objetivo Específicos}
%a preencher

\section{Metodologia}
%a preencher

\section{Estrutura do Trabalho}
Na Seção \ref{desenvolvimento}.

%Engenharia de Software, Agil, Scrum, Planning Poker, IA, Engenharia de Requisitos, Engenharia de Requisitos para IA


Este trabalho está organizado em cinco Capítulos, além deste, consistindo em:
\begin{itemize}

    \item \textbf{Capítulo 2:} apresenta a fundamentação teórica em relação aos conceitos de ética em IA. Além disso, são apresentados os trabalhos correlatos identificados na revisão de literatura.

    \item \textbf{Capítulo 3:} apresenta uma proposta de um guia para apoiar a implementação da ética no desenvolvimento de aplicações no contexto de IA.

    \item \textbf{Capítulo 4:} apresenta os resultados da validação do guia proposto.

    \item \textbf{Capítulo 5:} apresenta as principais conclusões deste trabalho e trata dos trabalhos futuros.

\end{itemize}